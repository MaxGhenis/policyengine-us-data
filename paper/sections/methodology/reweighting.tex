\subsection{Reweighting Procedure}

We optimize household weights using gradient descent through PyTorch \citep{pytorch2019}.

\subsubsection{Problem Formulation}

Given a loss matrix $M$ of household characteristics and a target vector $t$, we optimize the log-transformed weights $w$ to minimize:

\[ L(w) = \text{mean}\left(\left(\frac{w^T M + 1}{t + 1} - 1\right)^2\right) \]

where:
\begin{itemize}
    \item $w$ are the log-transformed weights (requires grad=True)
    \item $M$ is the loss matrix in tensor form (float32)
    \item $t$ are the targets in tensor form (float32)
\end{itemize}

\subsubsection{Optimization Implementation}

The procedure follows these steps:

\begin{enumerate}
    \item Initialize with log-transformed original weights
    \item Create a PyTorch session with retries for robustness
    \item Use Adam optimizer with learning rate 0.1
    \item Apply dropout (5\% rate) during optimization
    \item Run for 5,000 iterations or until convergence
\end{enumerate}

\subsubsection{Dropout Application}

The dropout process:
\begin{itemize}
    \item Randomly masks p\% of weights each iteration (p = 5)
    \item Replaces masked weights with mean of unmasked weights
    \item Returns original weights if dropout rate is 0
\end{itemize}

\subsubsection{Convergence Monitoring}

For each iteration:
\begin{itemize}
    \item Track initial loss value as baseline
    \item Compute relative change from starting loss
    \item Display progress with current loss values
\end{itemize}

\subsubsection{Error Handling}

The implementation includes checks for:
\begin{itemize}
    \item NaN values in weights
    \item NaN values in loss matrix
    \item NaN values in loss computation
    \item NaN values in relative error calculation
\end{itemize}

If any check fails, the procedure raises a ValueError with diagnostic information.

\subsubsection{Weight Recovery}

The final weights are recovered by:
\begin{itemize}
    \item Taking exponential of optimized log weights
    \item Converting from torch tensor to numpy array
\end{itemize}