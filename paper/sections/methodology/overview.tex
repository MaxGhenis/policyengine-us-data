\section{Methodology}\label{sec:methodology}

Following \cite{bryant2022}, our procedure enhances the Current Population Survey (CPS) with tax information from the Public Use File (PUF) through four key steps:
\begin{enumerate}
    \item Project both CPS and PUF data to the target year
    \item Transfer tax variable distributions from PUF to CPS records
    \item Generate dependent age, primary age, and earnings split variables
    \item Reweight households to match administrative benchmarks
\end{enumerate}

\subsection{Data Projection}

We project both datasets forward using variable-specific growth indices $f(t)$:

\[ y_{2024} = y_{2023} \cdot \frac{f(2024)}{f(2023)} \]

The indices come from:
\begin{itemize}
    \item CBO economic projections for income components
    \item SSA wage index forecasts for employment income
    \item Census population projections for demographics
    \item Treasury forecasts for tax variables
\end{itemize}

For the PUF, we first age the 2015 data to 2021 using IRS Statistics of Income data, then project to 2024 using the same indices.

\subsection{Demographic Variable Construction}

Following \cite{bryant2022}, we construct several key demographic variables:

\subsubsection{Dependent Ages}
We create three dependent age variables (AGEDP1/2/3) capturing:
\begin{itemize}
    \item Up to 3 dependents for joint/HOH returns
    \item Up to 2 dependents for single returns 
    \item Up to 1 dependent for MFS returns
\end{itemize}

Ages are categorized as:
\begin{itemize}
    \item Under 5
    \item 5 under 13
    \item 13 under 17
    \item 17 under 19
    \item 19 under 24
    \item 24 or older
\end{itemize}

Dependents are ordered sequentially by type:
\begin{enumerate}
    \item Children living at home
    \item Children living away from home
    \item Other dependents
    \item Parents
\end{enumerate}

\subsubsection{Primary Taxpayer Age}
We construct age ranges differently for:

Non-dependent returns:
\begin{itemize}
    \item Under 26
    \item 26 under 35
    \item 35 under 45
    \item 45 under 55
    \item 55 under 65
    \item 65 or older
\end{itemize}

Dependent returns:
\begin{itemize}
    \item Under 18
    \item 18 under 26
    \item 26 or older
\end{itemize}

\subsubsection{Earnings Splits}
For joint returns, we calculate earnings splits using:
\[ \text{Primary Share} = \frac{\text{Primary Wages} + \text{Primary SE Income}}{\text{Total Wages} + \text{Total SE Income}} \]

Where:
\begin{itemize}
    \item Primary wages and SE income = E30400 - E30500
    \item Secondary wages and SE income = E30500
\end{itemize}

We categorize the splits as:
\begin{itemize}
    \item 75 percent or more earned by primary
    \item Less than 75 percent but more than 25 percent earned by primary
    \item Less than 25 percent earned by primary
\end{itemize}

\subsection{Tax Variable Enhancement}

We transfer tax variables from PUF to CPS using quantile regression forests trained on:
\begin{itemize}
    \item Constructed demographic variables described above
    \item Filing status
    \item Existing income measures
\end{itemize}

For each variable, we:
\begin{enumerate}
    \item Train a forest on PUF records
    \item Generate predicted distributions for CPS records
    \item Sample preserving rank within demographic groups
\end{enumerate}

\subsection{Household Reweighting}

We adjust household weights to minimize discrepancies with administrative targets while avoiding overfitting:

\[ \min_w \sum_j \left(\frac{\sum_i w_i x_{ij} - t_j}{t_j}\right)^2 + \lambda \sum_i (w_i - w_i^0)^2 \]

subject to:
\[ w_i \geq 0 \quad \forall i \]

where:
\begin{itemize}
    \item $w_i$ is the new weight for household $i$
    \item $w_i^0$ is the original CPS weight
    \item $x_{ij}$ is the value of variable $j$ for household $i$
    \item $t_j$ is the administrative target for variable $j$
    \item $\lambda$ controls regularization strength
\end{itemize}

We solve using gradient descent with 5\% dropout for regularization.