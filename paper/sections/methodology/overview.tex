\section{Methodology}

Our procedure transforms the Current Population Survey (CPS) into an enhanced microsimulation dataset through four key steps:
\begin{enumerate}
    \item Project both CPS and PUF data to the target year
    \item Transfer tax variable distributions from PUF to CPS records
    \item Impute program participation
    \item Reweight households to match administrative benchmarks
\end{enumerate}

\subsection{Data Projection}

We project the CPS forward using a combination of economic and demographic factors. For each economic variable $y$, we apply:

\[ y_{2024} = y_{2023} \cdot \frac{f(2024)}{f(2023)} \]

where $f(t)$ represents the variable-specific growth index. We derive these indices from:
\begin{itemize}
    \item CBO economic projections for aggregate income components
    \item SSA wage index forecasts for employment income
    \item Census population projections for demographic totals
    \item Treasury forecasts for tax variables
\end{itemize}

For the PUF, we first age the 2015 data to 2021 using IRS Statistics of Income data, then project to 2024 using the same indices as the CPS projection.

\subsection{Tax Variable Enhancement}

We transfer 47 tax variables from the PUF to the CPS using quantile regression forests. For each variable, we:
\begin{enumerate}
    \item Train a forest on PUF records using age, sex, marital status, and existing income measures as predictors
    \item Generate a distribution of predicted values for each CPS record
    \item Sample from these distributions using rank preservation within demographic groups
\end{enumerate}

This approach preserves both the marginal distributions of tax variables and their relationships with demographic characteristics. 

\subsection{Program Participation}

We model participation in major benefit programs through a two-stage process:
\begin{enumerate}
    \item Calculate eligibility using program rules
    \item Assign participation probabilities based on:
        \begin{itemize}
            \item Demographic characteristics
            \item Benefit amounts
            \item Geographic patterns
            \item Historical take-up rates
        \end{itemize}
\end{enumerate}

The final participation patterns emerge from our reweighting procedure's alignment with administrative totals.

\subsection{Household Reweighting}

We adjust household weights to minimize discrepancies with administrative benchmarks while avoiding overfitting. The optimization problem takes the form:

\[ \min_w \sum_j \left(\frac{\sum_i w_i x_{ij} - t_j}{t_j}\right)^2 + \lambda \sum_i (w_i - w_i^0)^2 \]

subject to:
\[ w_i \geq 0 \quad \forall i \]

where:
\begin{itemize}
    \item $w_i$ is the new weight for household $i$
    \item $w_i^0$ is the original CPS weight
    \item $x_{ij}$ is the value of variable $j$ for household $i$
    \item $t_j$ is the administrative target for variable $j$
    \item $\lambda$ controls the strength of regularization
\end{itemize}

We solve this using gradient descent with dropout, randomly zeroing 5\% of household weights during each iteration to improve generalization.

The remainder of the methodology section details each component:
\begin{itemize}
    \item Section 4.1 describes our quantile regression forest implementation
    \item Section 4.2 explains the reweighting optimization
    \item Section 4.3 presents our validation framework
\end{itemize}