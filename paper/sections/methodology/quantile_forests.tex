\section{Quantile Regression Forests}\label{sec:qrf}

We use quantile regression forests (QRF) in two distinct ways: direct imputation of missing variables, and generation of synthetic records.

\subsection{PUF Integration: Synthetic Record Generation}

Unlike our other QRF applications, we use the PUF to generate an entire synthetic CPS-structured dataset:

\begin{enumerate}
    \item Train QRF models on PUF records with demographic variables
    \item Generate a complete set of synthetic CPS-structured records using PUF tax information
    \item Stack these synthetic records alongside the original CPS records
    \item Allow the reweighting procedure to determine optimal mixing between CPS and PUF-based records
\end{enumerate}

This approach preserves CPS's person-level detail crucial for modeling:
\begin{itemize}
    \item State tax policies
    \item Benefit program eligibility
    \item Age-dependent federal provisions (e.g., Child Tax Credit variations by child age)
    \item Family structure interactions
\end{itemize}

\subsection{Direct Variable Imputation}

For other enhancement needs, we use QRF to directly impute missing variables:

\subsubsection{Housing Costs from ACS}
We impute rent payments and property taxes using ACS records, with predictors including:
\begin{itemize}
    \item Household head status
    \item Age
    \item Sex
    \item Tenure type
    \item Employment income
    \item Self-employment income
    \item Social Security income
    \item Pension income
    \item State
    \item Household size
\end{itemize}

\subsubsection{Prior Year Income from CPS ASEC Panel}
To support analysis of lookback provisions, we impute prior year earnings using consecutive-year ASEC records, using:
\begin{itemize}
    \item Employment income
    \item Self-employment income
    \item Household weight
    \item Income imputation flags
\end{itemize}

\subsection{Implementation Details}

Our QRF implementation in utils/qrf.py handles:
\begin{itemize}
    \item Categorical variable encoding
    \item Consistent feature ordering
    \item Distribution sampling
    \item Model persistence
\end{itemize}

% TODO: Add specifics about:
% - QRF hyperparameters
% - Computational performance
% - Validation metrics for both synthetic record generation and direct imputation
% - Details on how the reweighting procedure balances CPS vs PUF records